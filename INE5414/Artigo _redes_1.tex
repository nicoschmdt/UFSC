\documentclass[article,12pt]{abntex2}
\usepackage[num]{abntex2cite}
\usepackage{booktabs}

\title{Arquitetura e Machine Learning em Segurança de IoT}
\date{Outubro 2019}
\author{Nicole Schmidt - 18203344}
\begin{document}
\maketitle
\textbf{Titulo:}

Arquitetura e Machine Learning em Segurança de IoT

\textbf{Resumo:}
\newline
A Internet das Coisas (IoT) é um conceito que se refere à interconexão digital de objetos cotidianos com a internet. Percebemos, cada vez mais, dispositivos IoT adentrando a sociedade e, frequentemente, ajudando e auxiliando o dia a dia das pessoas. Entretanto, a vasta área que o IoT acaba englobando abre uma vasta possibilidades de ataques os quais visam roubar informações para, geralmente, uso maléfico. Neste artigo será discutido diversas propostas recentes sobre arquitetura de segurança e como usar machine learning para aprimorar as técnicas de segurança utilizadas atualmente.
\section{Introdução}
\subsection{Motivação}
Apesar dos avanços e pesquisas em redes IoT,  no quesito de segurança ainda é um problema presente nesse campo. Não obstante, o crescente aumento do uso de dispositivos IoT e sua grande dispersão em diversos meios de nossa sociedade põe em risco diversos fatores de nosso dia a dia. Os dispositivos IoT acabam sendo vítimas de diversos ataques, como por exemplo o ataque DDoS Mirai que capturou milhões de dispositivos IoT e transformou-os em botnets, os quais poderiam ser usados em ataques DDoS. Dessa forma, para garantir a segurança, tanto dos usuários quanto das empresas. 
%arrumar essa parte da motivação cuz geezus
\subsection{Justificativas}
O tópico de IoT vem crescendo rapidamente no mercado, e com isso também cresce a necessidade de impedir ataques à esses dispositivos, é necessário que os métodos utilizados não prejudiquem o sistema, o qual já não possui uma capacidade muito alta para lidar com programas pesados, sendo dessa forma simples e efetivo. Assim, a utilização de uma arquitetura de segurança e de métodos eficientes tornam-se essenciais para a proteção do dispositivo. Uma tática que vem sendo recentemente explorada é a utilização de machine learning para identificar ataques cibernéticos.
\subsection{Objetivos}
\subsubsection{Objetivos Gerais}
Apresentar artigos que falam sobre arquitetura e machine learning em segurança de dispositivos IoT
\subsubsection{Objetivos Específicos}
\begin{itemize}
    \item Discutir a proposta de implementação de algumas arquiteturas de segurança de IoT
    \item Mostrar como o machine learning pode ser utilizado na área de segurança em IoT
\end{itemize}
\subsection{Organização do Artigo}
O artigo será apresentando da seguinte forma: Na seção 2 – Conceitos Básicos: a definição e explicação de IoT e métodos de machine learning, além de uma breve explicação de oque seria uma arquitetura de segurança em IoT. Na seção 3– Trabalhos Correlatos: irá ser apresentado artigos relacionados à arquiteturas de segurança e métodos de utilização de machine learning para evitar ataques cibernéticos à dispositivos IoT. Na seção 4 - Aspectos Relevantes: é apresentado diversas propostas e suas diversificações em relação a grande gama de diferentes arquiteturas para dispositivos IoT existentes. Na seção 5 - Problemas Existentes: são apresentados alguns problemas que demonstram ser prejudiciais na busca por uma arquitetura para IoT que seja eficiente além de comentar sobre o problema relacionado à armazenamento em dispositivos IoT. Na seção 6 - Soluções Possíveis: Na seção 7 - Projeto e desenvolvimento de uma proposta: . Na seção 8 - Conclusões e Trabalhos Futuros:
%Na seção 6 – Soluções Possíveis -, sãodemonstrados técnicas e algoritmos capazes de detectar variados tipos de ataques de negaçãode serviço. Na seção 7 – Projeto e desenvolvimento de uma proposta – é demonstrado umalgoritmo  capaz de detectar com eficiência uma ataque DDoS. Na seção 8 – Conclusões eTrabalhos   Futuros   –   são   apresentadas   conclusões   gerais   sobre   o   artigo   elaborado   emencionado possíveis trabalhos futuros. 
\section{Conceitos básicos}
\subsection{O que é IoT}
A IoT, ou Internet das Coisas, refere-se à interconexão de dispositivos conectados à internet que estão presentes no nosso dia-a-dia. Ela é uma rede de objetos físicos capaz de reunir e de transmitir dados e uma extensão da internet que possibilita objetos do cotidiano a conectarem-se à internet, sendo composta apenas por sensores e outros dispositivos inteligentes. A internet das coisas pode ser aplicada em diversos setores, seja para otimizar as atividades de uma indústria ou facilitar a vida dos cidadãos. Neste artigo veremos métodos e práticas de segurança utilizadas atualmente as quais possuem o intuito de proteger os dados produzidos pelos dispositivos IoT e, dessa forma, proteger a privacidade dos usuários.
\subsection{O que é Machine Learning}
Machine learning é um método de análise de dados que automatiza a construção de modelos analíticos, é baseado na ideia de que sistemas podem aprender com dados, identificar padrões e tomar decisões com o mínimo de intervenção humana. Existem dois tipos de machine learning, a aprendizagem supervisionada, a qual existe um conjunto prévio de dados inseridos na máquina e as sugestões que serão dadas ao usuário devem ser parecidas com os dados registrados, e a aprendizagem não supervisionada, na qual não existe um resultado específico esperado. Neste artigo daremos enfase a métodos que utilizam machine learning para a detecção de anomalias que podem ser, possivelmente, ataques cibernéticos.
\subsection{O que seria uma arquitetura de segurança em IoT}
Uma arquitetura descreve a estrutura da solução de IoT, incluindo os aspectos físicos (no caso, as coisas) e os aspectos virtuais (como serviços e protocolos de comunicação). Desenvolver uma solução end-to-end (E2E) segura envolve em múltiplos leveis que agregam importantes características de uma arquitetura de segurança IoT em diferentes camadas. Procura-se garantir que a informação esta seguramente criptografada enquanto está transitando, ou mesmo parada, para que assim, mesmo que ela seja interceptada não seja possível obter seu conteúdo.
\section{Trabalhos Correlatos}
\textbf{Revisão bibliográfica sistemática:}\\
\begin{tabular}{l l}
    \toprule
    \textbf{Palavra-chave}& \textbf{Total}  \\
    \midrule
    \textbf{``Machine Learning''}& \textbf{ 5.360.000 }\\
    \textbf{``IoT''}& \textbf{ 892.000 } \\
    \textbf{``IoT'' ``Machine Learning''}& \textbf{ 67.100 }\\
    \textbf{``IoT'' ``Machine Learning'' ``Security''}& \textbf{ 45.800 } \\
    \textbf{``IoT'' ``Machine Learning'' ``Security Architecture''}& \textbf{ 38.600 } \\
    \bottomrule
\end{tabular}

É notável, através da análise da tabela acima, que ao se aprofundar na área de pesquisa em segurança de IoT e no uso de machine learning para o auxiliar com tal objetivo há um decréscimo no número de artigos devido à especificidade do tema, no entanto é possível perceber que mesmo assim este tópico já possui certa importância para a área, sendo 5\% do total de artigos de IoT pertencendo a esse tema e 57.5\% de artigos no tema de IoT e Machine Learning. Agora serão analisados alguns artigos referenciados na seção bibliográficas.
\subsection{System Statistics Learning-Based IoT Security: Feasibility and Suitability}%Li - 2019
O artigo possui como foco os problemas de segurança IoT em serviços fog, que são baseados em dispositivos com capacidade computacional e de comunicação limitada. Os autores propõem uma framework de detecção geral de anomalias a qual possui o intuito de detectar possíveis ataques cibernéticos ou atividades maliciosas. A técnica utilizada consiste em ver estatísticas do simples do sistema como a utilização da CPU, memoria consumida, etc. para definir um comportamento normal. É mostrado que técnicas estatísticas e a utilização de modelos de machine learning podem ser usadas efetivamente para detectar ataques em dispositivos IoT. O emprego de modelos de machine learning é utilizado para prever o comportamento dos dispositivos, através  de métodos de analise de series temporais, e observar o desvio que eles acabam seguindo ao invés de executarem como o esperado. Esse método também pode ser utilizado para detectar falhas pois ele não foi modelado para detecta-las e portanto o comportamento anômalo pode ser identificado.
\subsection{A Security Architecture for RISC-V based IoT Devices}%2019-auer
No artigo os autores explicam que a arquitetura apresentada foca em três áreas: o secure boot (boot seguro), a autenticação watch dog e a gestão de chaves. O secure boot é utilizado devido à necessidade de verificar se o update de um software compromete o dispositivo de alguma forma. Essa funcionalidade começa a ser executada a partir da inicialização para autenticar a imagem de inicialização, atualizações são feitas durante a execução, um sistema de gerenciamento centralizado deve ser capaz de realizar essa tarefa remotamente além de monitorar esse a fim de detectar falhas. A autenticação watch dog reseta uma vez que o seu timer interno encerra. No entanto, uma vez que ativado ele só pode ser desativado por um mecanismo externo. Após isso o secure boot então é capaz de restaurar o dispositivo para uma versão confiável, isso possibilita que os dispositivos sejam capazes de serem recuperados mesmo que eles não estejam conectados à uma rede. Em redes muito grandes tais verificações necessitam ser feitas automaticamente. A gestão de chaves foi feita com a intenção de manter a segurança dos dados, elas vem da identidade DICE através de um algoritmo de derivação de chaves, a identidade é baseada na chave criptográfica única existente no chip. Dessa forma, toda vez em que ocorra uma restauração de fábrica uma nova identidade seja criada, isso garante que todas as chaves criptográficas utilizadas pelo sistema também serão alteradas.

\subsection{IoTChain: A blockchain security architecture for the Internet of Things}%alphand - 2018
Os autores propõem  uma combinação entre a arquitetura OSCAR (Object Security Architecture), a qual provém uma criptografia E2E (end to end) para o transporte de informações de dispositivos IoT, e a estrutura de autorização ACE (Authentication and Authorization for Constrained Environments),aonde os tokens são fornecidos por um servidor de autorização. É proposto que a única verificação de autorização feita pela interface ACE seja substituída por uma blockchain não confiável, pois dessa forma o controle ao acesso de recursos seria robusto, flexível e possivelmente preservaria a privacidade do usuário, isso devido ao protocolo de consenso da blockchain que faz com que o atacante precise ter um controle de 51\% dela antes que ele consiga obter tokens ilegítimos. A blockchain lida com pedidos de autorização através de smart contracts (contratos inteligentes),os quais são definidos pelo dono do recurso oferecido, adicionando um token no contrato de armazenamento para autorizar um cliente. Além disso é utilizado um grupo de distribuição de chaves self-healing que permite uma eficiente transmissão de recursos IoT para múltiplos destinatários simultaneamente. Os clientes solicitam chaves do servidos de chaves para se juntar ao grupo de distribuição de chaves associado com os recursos desejados.

\subsection{SDN Enabled Secure IoT Architecture}%karmakar - 2019
No artigo, a qual propõe uma nova arquitetura de segurança para dispositivos IoT, os autores explicam o porquê de escolherem trabalhar com Software-Defined Network(SDN) no quesito de segurança, principalmente devido à separação do plano de controle do plano de dados, ao domínio da rede e as aplicações SDN northbound. O SDN permite que apenas informações autenticadas de um dispositivo IoT legitimado seja capaz de descobrir e usar os serviços de uma rede. Para isso, foram desenvolvidas duas aplicações que controlam e gerenciam o comportamento dos dispositivos na rede, o primeiro é o policy based security application (PbSA) e o segundo o IoT Security Applications (ISA), o qual possui dois sub-módulos, o IoT Authorization Authority e o IoT Authentication Authority. Nessa arquitetura cada dispositivo possui um ID único que conecta a um IoT gateway (dispositivo que intermediará o fluxo de dados entre os sensores e o data center) através de uma rede wireless, os dispositivos são autenticados na primeira fase do protocolo de segurança, e então o dispositivo manda uma mensagem com um requerimento para um serviço de rede, no final dessa fase é estabelecido uma chave comum secreta com o aparelho a qual pode ser utilizada posteriormente para próximas comunicações. Na etapa de autorização é conferido com a chave gerada  o serviço de rede o qual o dispositivo fez o requerimento, se o dispositivo e se os seus atributos de fluxo satisfazem as expressões definidas pelo IoT Authorization Authority que então irá gerar um token para o dispositivo IoT acessar o serviço requisitado.

\section{Aspectos Relevantes}
%ver artigos escolhidos 
Devido à imensa variedade de dispositivos IoT é compreensível que exista uma diversa gama de propostas de arquitetura diferentes dependendo das características que o dispositivo apresenta, do microcontrolador, etc. Podemos notar que em \cite{Das:2018} é apresentado uma forma de autenticação voltada para que dispositivos de baixa potencia consigam se defender de dispositivos mais potentes usando uma técnica de machine learning, a qual irá dinamicamente aprender as imperfeições dos transmissores do hardware através de canais wireless observados. De outra forma, em \cite{Minoli:2017} é proposto um modelo de referencia OSI (OSiRM) mais detalhado e otimizado especificamente para cada camada, apesar do objetivo seja mais do tipo: dependendo do tipo de aplicação teremos uma implementação de segurança diferente em cada layer, também é incluído um mecanismo de segurança que pode existir independentemente em cada layer e, caso necessário, é possível incluir mais mecanismos. Nesse último caso é perceptível que ocorrerá uma maior demanda de eficiência energética, todavia, várias propostas atuais buscam achar formas para solucionar essa problemática.
\newline
No artigo \cite{Oliveira:2018} é proposto a ideia de construir uma arquitetura baseada em um microchip reconfigurável, o qual integraria uma unidade de microcontrolador incondicional além de uma unidade computacional reconfigurável (RCU), é dito que ela permitiria o desenvolvimento the aceleradores de hardware customizados. Dentre as propostas para a construção de tal arquitetura é comentado que, devido a escassez de soluções existentes com suporte conectivo é necessário que uma interface de rádio seja anexada aos nós sensores. Apesar de isso não ser uma proposta implementada ela aborda um fator interessante pois além de ela ir atrás de uma boa gestão no quesito energético essa configuração permite que o sistema operacional explore a isolação TrustZone à favor da segurança. De outra forma, \cite{Novo:2018} explora a utilização da tecnologia segura, decentralizada e autônoma de uma blockchain. Apesar de ser utilizada, os dispositivos, principalmente por uma questão de capacidade de armazenamento, não estão incluídos na blockchain, alternativamente é definido um novo nó chamado de management hub o qual requisita acesso de controle de informação da blockchain em nome dos dispositivos IoT. Essa implementação é atingida através de um único smart-contract que define todas as operações permitidas no controle de acesso do sistema. O management hub não utiliza uma transação para pegar a informação armazenada na blockchain, ela é buscada diretamente do nó. A principio a blockchain seria armazenada em uma nuvem para que, desta forma, os dispositivos não fiquem sobrecarregados em ter que armazenar todo o conteúdo existente da blockchain, assim é possível que tenham-se milhares de dispositivos conectados à essa blockchain sem que haja algum problema relacionado com capacidade de armazenamento.

\section{Problemas Existentes}
O crescente aumento de dispositivos IoT conectados à internet trouxe uma gama de dificuldades e desafios. A imposição de novas exigências, tais como um armazenamento mais eficiente, afinal os dados precisam ir para algum lugar para que se tornem úteis, o aumento do tráfego para as redes, a demanda de processamento de dados acabam por pressionar uma infraestrutura mais eficiente. No entanto, a tendência de cada vez mais se produzirem mais e mais dados pode acabar com a ideia de reduzir custos e aumentar a segurança, tornando-a insustentável. Algumas medidas tomadas, como por exemplo o proposto por  \cite{Stergiou:2017}, a qual visa prover uma arquitetura segura e com qualidade de comunicação, mas com isso acaba-se tendo uma desvantagem em relação à velocidade de transmissão e eficiência.
\newline
Atualmente não existe um padrão internacional de compatibilidade para identificação e monitoramento de dispositivos, isso acarreta em uma dificuldade para criar uma arquitetura única que atingiria todos os padrões e requisitos desejados por cada dispositivo existente conectado à internet. A técnica vista em \cite{Li:2019} demonstra englobar mais diferentes tipos de hardware de IoT já existentes, no entanto eles possuem um foco voltado a aumentar a segurança em fog computing, o que acaba diminuindo o range de quais dispositivos IoT poderiam usufruir de tal pesquisa. Infelizmente, pensar em uma estrutura que consiga abranger todos os dispositivos conectados aparenta ser inviável, afinal isso envolveria que todos os envolvidos nesse processo adotem essa arquitetura.

\section{Soluções Possíveis}
A diversidade de arquiteturas de dispositivos IoT presentes na contemporaneidade e a dificuldade de criar um padrão de segurança para ser seguido por elas trás consigo um desafio de dar uma única solução que proteja de forma eficiente todos os dispositivos existentes até o momento. Aqui serão discutidas algumas soluções apresentadas pelos trabalhos correlatos que procuram solucionar uma pequena gama de problemas de segurança em certos tipos de dispositivos.
\newline
Seguindo pelo a ideia de uma blockchain, isto é, uma base de dados distribuída na qual não é possível confiar em todos os participantes é que seja pública, ou seja, qualquer um  pode acessa-lá, um problema de seu uso é que todos os participantes de uma blockchain possuem à própria blockchain completa consigo acarentando em uma boa necessidade de armazenamento para sua utilização, uma forma que \cite{Alphand:2018} encontrou para evitar esse problema é a utilização é utilizando a blockchain somente para que o controle de acesso, como a blockchain possui um protocolo de consenso isso faria com que algum atacante precisasse ter controle de 51\% da blockchain para causar algum dano ou prejuízo ao dispositivo.
\newline
Diferente de \cite{Alphand:2018}, \cite{Auer:2019} utiliza um watchdog timer o qual dispara um reset no sistema caso o programa deixe de fazer o reset no watchdog, é utilizado para detectar e restaurar o dispositivo de algum mau funcionamento do computador. Uma vez que o watchdog é ativado ele só pode ser desativado por um mecanismo externo, podendo evitar assim que algo que possa ativa-ló não o desative. Caso isso aconteça é chamado um secure boot externo que irá restaurar o dispositivo para uma versão confiável, evitando assim que a causa de um possível mau funcionamento seja por alguma alteração no hardware/software ou de alguma perda no controle daquele dispositivo. É preciso cuidar pois em redes muito grandes essas verificações podem necessitar de uma automatização.
\section{Projeto e Desenvolvimento de uma Proposta}
%Esse é so na 3 entrega..
\section{Conclusões e Trabalhos Futuros}
\subsection{Conclusões}
A internet das coisas irá, cada vez mais, fazer parte do nosso cotidiano. Para que isso não acabe sendo um risco para a nossa privacidade e segurança é necessário que medidas e precauções sejam tomadas junto do avanço dessa tecnologia.
\newline
Apesar de termos diversos fatores que colaboram para termos dispositivos IoT mais seguros este artigo escolheu analisar as abordagens mais atuais voltadas para a arquitetura e as tecnologias que estão sendo aplicadas nesta área. Sendo assim, vimos que existe diferentes arquiteturas para cada tipo de objetivo, caso o aplicativo possua um limite de armazenamento, que é geralmente o caso, existem diversas possibilidades de escolha, é notável que devido a grande gama de arquiteturas existentes a falta de um padrão internacional de compatibilidade, oque acaba dificultando a criação de uma arquitetura que alcance todos os padrões e requisitos desejados.
\subsection{Trabalhos Futuros}
Conforme a criação de novas tecnologias e métodos é natural que com esse avanço também sejam criadas novos meios de ataque ou uma possível potencialização de ataques já existentes. Considerando que o estado da arte do IoT está em constante mudança é imprescindível que os estudos na área de segurança, incluindo em sua arquitetura, sejam aprofundados, aumentando a segurança e confiabilidade dos usuários em tal tecnologia que pode, de diversas formas, melhorar o dia-a-dia da população. Desta forma, novas técnicas de segurança aplicadas em arquitetura podem minimizar o dano de ataques já existentes como o DDoS, que como foi visto, pode-se diminuir a eficiencia desse ataque utilizando uma blockchain, ou detectando alguma alteração nele e reseta-lo à um estado de confiabilidade utilizando um watchdog.


\bibliography{refs}
\nocite{Auer:2019}
\nocite{Tiburski:2019}
\nocite{Li:2019}
\nocite{Karmakar:2019}
\nocite{Oliveira:2018}
\nocite{Das:2018}
\nocite{Alphand:2018}
\nocite{Novo:2018}
\nocite{Stergiou:2017}
\nocite{Minoli:2017}

\end{document}
